\section{Introdução}

Considere a matriz

\begin{equation}
\label{cap5:sec1:eq1}
 A =
 \left[
 \begin{array}{rrr}
  16 & -24 & 18\\
  3  & -2  & 0\\
  -9 & 18  & -17
 \end{array}
 \right]
\end{equation}

Em geral, para vetores $ \{x\} \in \varmathbb{R}^3 $

\begin{equation}
\label{cap5:sec1:eq2}
[A] \, \{x\} = \{y\}
\end{equation}

onde $ \{y\} \neq \{x\} $ e $ \{y\} \in \varmathbb{R}^3 $.

De todos os vetores $ \{x\} \in \varmathbb{R}^3 $ existem vetores $ \{x^*\} $ tal que

\begin{equation}
\label{cap5:sec1:eq3}
[A] \, \{x^*\} = \lambda \, \{x^*\}
\end{equation}

ou seja, $ \{y\} $ é proporcional a $ \{x^*\} $.

\begin{definicao}
Os vetores $ \{x\} $ que satisfazem \ref{cap5:sec1:eq3} são chamadas \textbf{auto-vetores} de $[A]$.
\end{definicao}

\begin{definicao}
As constantes de proporcionalidade, $ \lambda $, são chamdas de \textbf{auto-valores}.
\end{definicao}

Por exemplo,

\begin{equation}
\label{cap5:sec1:eq4}
 \left[
  \begin{array}{rrr}
   16 & -24 & 18\\
   3  & -2  & 0\\
   -9 & -18 & -17
  \end{array}
 \right]
 \,
 \left\{
  \begin{array}{c}
   2\\
   1\\
   0
  \end{array}
 \right\}
 =
 4 \,
 \left\{
  \begin{array}{c}
   2\\
   1\\
   0
  \end{array}
 \right\}
\end{equation}

\textbf{OBS:} Para que $ \{x\} $ e $ \{y\} $ pertençam ao mesmo espaço vetorial, a matriz $ A $ tem que ser quadrada. Assim, só matrizes quadradas possuem auto-valores e auto-vetores.

Como encontrar os auto-valores e auto-vetores de $ A $?

\begin{equation}
\label{cap5:sec1:eq5}
 \begin{array}{lll}
              & A \, x = \lambda \, x &\\
  \Rightarrow & A \, x = \lambda \, I \, x &\\
  \Rightarrow & [A - \lambda \, I] \, \{x\} = \{0\} & \mbox{(Sist. de Eq. Homogêneos)}
 \end{array}
\end{equation}

onde $ [A - \lambda \, I] $ é a matriz característica.

\begin{equation}
\label{cap5:sec1:eq6}
 \left[
 \begin{array}{rrr}
  16 - \lambda & -24          & 18\\
  3            & -2 - \lambda & 0\\
  -9           & 18           & -17-x
 \end{array}
 \right]
 \,
 \left\{
 \begin{array}{c}
  x_1\\
  x_2\\
  x_3
 \end{array}
 \right\}
 =
 \left\{
 \begin{array}{c}
  0\\
  0\\
  0
 \end{array}
 \right\}
\end{equation}

Se $ det \, [A - \lambda \, I] \neq 0 $ a solução do sistema é $ \{x\} = \{0\} $. Para que $ \{x\} \neq \{0\} $, então:

\begin{equation}
\label{cap5:sec1:eq7}
 det \, [A - \lambda \, I] = 0
\end{equation}

Para a matriz $ [A - \lambda \, I] $ na equação \ref{cap5:sec1:eq6}, $ det \, [A - \lambda \, I] = 0 $ leva

\[
 \begin{array}{c}
  \lambda^3 + 3 \, \lambda^2 - 36 \, \lambda + 32 = 0\\
  \Updownarrow\\
  (\lambda - 4) \, (\lambda - 1) \, (\lambda + 8) = 0
 \end{array}
\]

Assim,

\equacao{ \lambda = 4 }

\equacao{ \lambda = 1 }

\equacao{ \lambda = -8 }

são auto-valores de $ [A] $.

Forma geral para matrizes de ordem $ n $:

\[
 \begin{array}{l}
  (A - \lambda \, I) \, q = 0\\
  det \, (A - \lambda \, I) = 0 \Leftrightarrow \lambda ^n + c_{n-1} \, \lambda^{n-1} + \ldots + c_1 \, \lambda + c_0 = 0
 \end{array}
\]

\subsection{Propriedades dos Auto-Valores}

\begin{enumerar}

\item a equação característica pode ser fatorada na forma

\[
 (\lambda - \lambda_1) \, (\lambda - \lambda_2) \ldots (\lambda - \lambda_n) = 0
\]

Assim, uma matriz de ordem $ n $ tem $ n $ auto-valores não necessariamente distintos.

\item A soma dos termos da diagonal de $ A $ é chamada de traço de $ A $. Assim,

\[
 tr(A) = a_{11} + a_{22} + \ldots + a_{nn} = - c_{n-1} = \lambda_1 + \lambda_2 + ... + \lambda_n
\]

\item $ det \, (A) = (-1)^n \, c_0 = \lambda_1 \, \lambda_2 \ldots \lambda_n $

Assim uma matriz singular deve ter pelo menos um auto-valor nulo.

\item Sabendo- se que $ det \, A = det \, A^T $, então os auto-valores de $ A $ e de $ A^T $ são idênticos.

\item A equação característica com coeficientes reais deve ter auto-valores reais ou um par de auto-valores conjugados complexos. ($ a \pm b \, i $)

\item Os auto-valores de uma matriz simétrica com coeficientes reais são reais.

\item Se $ A $ for triangular

\[
 det \, (A - \lambda \, I) = (a_{11} - \lambda) \, (a_{22} - \lambda) \ldots (a_{nn} - \lambda) = 0
\]

Assim, os termos da diagonal são auto-valores.

\item Trocas de linhas e colunas correspondentes não afetam os auto-valores da matriz.

\item Se a linha $ i $ de $ A $ é multiplicada por $ f $ e a coluna $ i $ de $ A $ é dividida por $ f $, os auto-valores não mudam.

\end{enumerar}
