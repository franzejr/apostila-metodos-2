\section{Método de Eliminação de Gauss Padrão com Pivotação}

\textbf{Notas:} O método da eliminação de Gauss não funciona se o primeiro coeficiente da primeira linha for zero ou um coeficiente da diagonal se tornar zero durante o processo.

\begin{itemize}
 \item Pivotação é usada para mudar a ordem seqüencial das linhas para:

\begin{itemize}
 \item Evitar que o coeficiente da diagonal se torne zero.

\item Trazer o maior coeficiente abaixo da diagonal para a diagonal.

\item A mudança da ordem das linhas (equações) não afetam a solução do sistema.

\item Mesmo que o termo da diagonal não seja zero, colocar um coeficiente dominante na diagonal melhora a precisão.
\end{itemize}

\end{itemize}

\begin{example}

\begin{equation}
 \label{cap4:sec2:eq1}
 \left[
 \begin{array}{rrr}
   0 & 10 & 1 \\
   1 & 3 & -1 \\
   2 & 4 & 1
 \end{array}
 \right]
 \,
 \left[
 \begin{array}{r}
   x_1 \\
   x_2 \\
   x_3
 \end{array}
 \right]
 =
 \left[
 \begin{array}{r}
   2 \\
   6 \\
   5
 \end{array}
 \right]
 \quad
 \begin{array}{l}
   \\
   L'_2 = L_2 - \frac{1}{0}\,L_1 \quad \mbox{(impossível)} \vspace*{0.05cm}\\
   L'_3 = L_3 - \frac{2}{0}\,L_1 \quad \mbox{(impossível)}
 \end{array}
\end{equation}

Troca linha 1 com linha 3:

\begin{equation}
 \label{cap4:sec2:eq2}
 \left[
 \begin{array}{rrr}
   2 & 4 & 1 \\
   1 & 3 & -1 \\
   0 & 10 & 1
 \end{array}
 \right]
 \,
 \left[
 \begin{array}{r}
   x_1 \\
   x_2 \\
   x_3
 \end{array}
 \right]
 =
 \left[
 \begin{array}{r}
   5 \\
   6 \\
   2
 \end{array}
 \right]
 \quad
 \begin{array}{l}
   L_1 \\
   L'_2 = L_2 - \frac{1}{2}\,L_1 \vspace*{0.05cm}\\
   L'_3 = L_3 - \frac{0}{2}\,L_1
 \end{array}
\end{equation}

\begin{equation}
 \label{cap4:sec2:eq3}
 \left[
 \begin{array}{rrr}
   2 & 4 & 1 \\
   0 & 1 & -3/2 \\
   0 & 10 & 1
 \end{array}
 \right]
 \,
 \left[
 \begin{array}{r}
   x_1 \\
   x_2 \\
   x_3
 \end{array}
 \right]
 =
 \left[
 \begin{array}{r}
   5 \\
   7/2 \\
   2
 \end{array}
 \right]
\end{equation}

Troca linha 2 com linha 3:

\begin{equation}
 \label{cap4:sec2:eq4}
 \left[
 \begin{array}{rrr}
   2 & 4 & 1 \\
   0 & 10 & 1 \\
   0 & 1 & -3/2
 \end{array}
 \right]
 \,
 \left[
 \begin{array}{r}
   x_1 \\
   x_2 \\
   x_3
 \end{array}
 \right]
 =
 \left[
 \begin{array}{r}
   5 \\
   2 \\
   7/2
 \end{array}
 \right]
 \quad
 \begin{array}{l}
   \\
   \\
   L'_3 = L_3 - \frac{1}{10}\,L_2
 \end{array}
\end{equation}

\begin{equation}
 \label{cap4:sec2:eq5}
 \left[
 \begin{array}{rrr}
   2 & 4 & 1 \\
   0 & 10 & 1 \\
   0 & 0 & -16/5
 \end{array}
 \right]
 \,
 \left[
 \begin{array}{r}
   x_1 \\
   x_2 \\
   x_3
 \end{array}
 \right]
 =
 \left[
 \begin{array}{r}
   5 \\
   2 \\
   33/5
 \end{array}
 \right]
\end{equation}

Retro-substituição:

\[
 x_3 = \frac{33}{5} : \frac{-16}{5} = - \frac{33}{5} \, \frac{5}{16} = -2.0625
\]

\[
 x_2 = \frac{1}{10} \, [2 - 1 \, (-2.0625)] = 0.40625
\]

\[
 x_1 = \frac{1}{2} \, [5 - (4\,0.40625 + 1\,(-2.0625))] = 2.7187
\]

Gauss-Jordan

\[
 \left[
 \begin{array}{rrr}
   2 & 0 & 3/5 \\
   0 & 10 & 1 \\
   0 & 0 & -16/5
 \end{array}
 \right]
 \,
 \left[
 \begin{array}{r}
   x_1 \\
   x_2 \\
   x_3
 \end{array}
 \right]
 =
 \left[
 \begin{array}{r}
   21/5 \\
   2 \\
   33/5
 \end{array}
 \right]
 \quad
 \begin{array}{l}
   L'_1 = L_1 - \frac{2}{5}\,L_2 \\
   \\
   \,
 \end{array}
\]

Divisão de $L_{3} : (-16/5)$

\[
 \left[
 \begin{array}{rrr}
   2 & 0 & 3/5 \\
   0 & 10 & 1 \\
   0 & 0 & 1
 \end{array}
 \right]
 \,
 \left[
 \begin{array}{r}
   x_1 \\
   x_2 \\
   x_3
 \end{array}
 \right]
 =
 \left[
 \begin{array}{r}
   21/5 \\
   2 \\
   -2.0625
 \end{array}
 \right]
 \quad
 \begin{array}{l}
   L'_1 = L_1 - \frac{3/5}{1}\,L_3 \\
   L'_2 = L_2 - \frac{1}{1}\,L_3 \\
   \,
 \end{array}
\]

\[
 \left[
 \begin{array}{rrr}
   2 & 0 & 0 \\
   0 & 10 & 0 \\
   0 & 0 & 1
 \end{array}
 \right]
 \,
 \left[
 \begin{array}{r}
   x_1 \\
   x_2 \\
   x_3
 \end{array}
 \right]
 =
 \left[
 \begin{array}{r}
   5.4375 \\
   4.0625 \\
   -2.0625
 \end{array}
 \right]
\]

\[
 \left[
 \begin{array}{rrr}
   1 & 0 & 0 \\
   0 & 1 & 0 \\
   0 & 0 & 1
 \end{array}
 \right]
 \,
 \left[
 \begin{array}{r}
   x_1 \\
   x_2 \\
   x_3
 \end{array}
 \right]
 =
 \left[
 \begin{array}{r}
   2.71875 \\
   0.40625 \\
   -2.06250
 \end{array}
 \right]
\]

\end{example}

\textbf{Nota:} Se a pivotação não impedir que apareça zero na diagonal, o problema é não resolvível.

\begin{example}

Exemplo real:

\[
\left[
 \begin{array}{llll}
  1.334 \cdot 10^{-4} & 4.123 \cdot 10^1 & 7.912 \cdot 10^2 & -1.544 \cdot 10^3 \\
  1.777 & 2.367 \cdot 10^{-5} & 2.070 \cdot 10^1 & -9.035 \cdot 10^1 \\
  9.188 & 0 & -1.0150 \cdot 10^1 & 1.988 \cdot 10^{-4} \\
  1.002 \cdot 10^2 & 1.442 \cdot 10^4 & -7.014 \cdot 10^2 & 5.321
 \end{array}
\right]
\,
\left[
 \begin{array}{llll}
  -711.5698662 \\
  -67.87297633 \\
  -0.961801200 \\
  13824.12100
 \end{array}
\right]
\]

\textbf{Solução:}

\begin{enumerar}

\item Precisão simples

{
\footnotesize
\centering
	
\begin{tabular}{|c|c|c|}
	\hline
	\textbf{i} & \textbf{$x_i$ \, (s/ pivotação)} & \textbf{$x_i$ \, (c/ pivotação)} \\
	\hline \hline
	$1$ & $0.95506$ & $0.99998$ \\
	\hline
	$2$ & $1.00816$ & $1.$ \\
	\hline
	$3$ & $0.96741$ & $1.$ \\
	\hline
	$4$ & $0.98352$ & $1.$ \\
	\hline
\end{tabular}
}

\item Precisão dupla

{
\footnotesize
\centering
	
\begin{tabular}{|c|c|c|}
	\hline
	\textbf{i} & \textbf{$x_i$ \, (s/ pivotação)} & \textbf{$x_i$ \, (c/ pivotação)} \\
	\hline \hline
	$1$ & $0.9999\,9999\,9801\,473$ & $1.0000\,0000\,0000\,002$ \\
	\hline
	$2$ & $1.0000\,0000\,0000\,784$ & $1.0000\,0000\,0000\,000$ \\
	\hline
	$3$ & $0.9999\,9999\,9984\,678$ & $1.0000\,0000\,0000\,000$ \\
	\hline
	$4$ & $0.9999\,9999\,9921\,696$ & $1.0000\,0000\,0000\,000$ \\
	\hline
\end{tabular}
}

\end{enumerar}

\end{example}


