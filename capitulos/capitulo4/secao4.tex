\section{Matrizes, Vetores e Inversão de Matrizes}

\begin{itemize}

\item Matrizes quadradas e retangulares.

\item Operações: adição, subtração, multiplicação e divisão

\begin{itemize}
 \item Divisão $B^{-1}A = C$ só existe se $B$ for quadrada.
\end{itemize}

\item Vetores

\item Produto de matriz por vetor

\end{itemize}

\subsection{Matriz Nula}

\[ A_{ij} = 0\]

\subsection{Matriz Identidade}

\[
  I =
  \left\{
    \begin{array}{ll}
      1, & \mbox{se $i = j$} \\
      0, & \mbox{se $i \neq j$}
    \end{array}
  \right.  %%% o ``.'' coloca um delimitador invisivel.
\]

\subsection{Matriz Transposta de A}

\[ A^{T} \Rightarrow a^{T}_{ij} = a_{ji} \]

\subsection{Matriz Inversa}

$A^{-1}$ de uma matriz quadrada

\[ AA^{-1} = A^{-1}A = I \]

Se $BA = I$ ou $AB = I$ então $B = A^{-1}$

\subsection{Matriz Ortogonal}

Matrizes cujas colunas são ortogonais entre si:

\[ Q^{T}Q = I;\,\, QQ^{T} = I;\,\, Q^{T} = Q^{-1} \]

\subsection{Vetor Nulo}

\[ a_{i} = 0 \]

\subsection{Vetor Unitário}

\[
  u = \left[
    \begin{array}{c}
      1 \\
      0 \\
      0 
    \end{array}
    \right];\,\,
  v = \left[
    \begin{array}{c}
      0 \\
      1 \\
      0 
    \end{array}
    \right]
  w = \left[
    \begin{array}{c}
      0 \\
      0 \\
      1 
    \end{array}
    \right]
\]

\[
u = \frac{\vec{u}}{|\vec{u}|}
\]

\subsection{Vetor Transposto}

\[
  v = \left[
    \begin{array}{c}
      x_1 \\
      x_2 \\
      x_3 
    \end{array}
    \right]
\,\,\,
  v^T = \left[
    \begin{array}{ccc}
      x_1 & x_2 & x_3 
    \end{array}
    \right]
\]

\subsection{Inversão de uma Matriz}

\[ Ax = y\]

Suponha que o processo de eliminação de Gauss-Jordan seja resumido na operação matricial.

\[ G \, A \, x = G \, y \]

Fazendo $G \, A = I$, temos:

\[ I \, x = G \, y \]

Assim

\begin{center}
$ G = A^{-1} \,$ ou $\, G \, I = A^{-1} $
\end{center}

\[
 \left[
  \begin{array}{ccc}
    a_{11} & a_{12} & a_{13}\\
    a_{21} & a_{22} & a_{23}\\
    a_{31} & a_{32} & a_{33}
  \end{array}
 \right]
  \,\,
\left[
  \begin{array}{ccc}
    1 & 0 & 0\\
    0 & 1 & 0\\
    0 & 0 & 1
  \end{array}
 \right]
\]

Aplicar Gauss-Jordan ou Gauss ``padrão'' (pivotação pode ser utilizada).

\[ [A][X] = [I] \]

\begin{example}

Calcule a inversa de

\[
 A =
  \left[
  \begin{array}{ccc}
    2 & 1 & -3\\
    -1 & 3 & 2\\
    3 & 1 & -3
  \end{array}
  \right]
\]

Resposta:

\[
 A^{-1} =
  \left[
  \begin{array}{ccc}
    -1 & 0 & 1\\
    0.27272 & 0.27272 & -0.09??0\\
    -0.90909 & 0.09090 & 0.63636
  \end{array}
  \right]
\]

\end{example}
