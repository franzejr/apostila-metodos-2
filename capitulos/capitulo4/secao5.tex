\section{Decomposição LU}

\[ A = L \, U \]

$L =$ matriz triangular inferior com diagonal 1

$U =$ matriz triangular superior com diagonal $\neq$ 1

\begin{center}
$ A \, x = y \Rightarrow L \, U \, x = y \,$
 ou
$
  \,
  \begin{array}{l}
    L \, z = y\\
    U \, x = z
  \end{array}
$
\end{center}

\[
 \left[
  \begin{array}{cccccc}
   a_{11} & a_{12} & \cdots & a_{1j} & \cdots & a_{1n} \\
   a_{21} & a_{22} & \cdots & a_{2j} & \cdots & a_{2n} \\
   \vdots & \vdots & \vdots & \vdots & \ddots & \vdots \\
   a_{i1} & a_{i2} & \cdots & a_{ij} & \cdots & a_{in} \\
   \vdots & \vdots & \vdots & \vdots & \ddots & \vdots \\
   a_{n1} & a_{n2} & \cdots & a_{nj} & \cdots & a_{nn}
  \end{array}
 \right]
 =
 \left[
  \begin{array}{cccc}
   1 & 0 & \cdots & 0 \\
   l_{21} & 1 & \cdots & 0 \\
   \vdots & \vdots & \ddots & \vdots \\
   l_{n1} & l_{n2} & \cdots & 1
  \end{array}
 \right]
 \left[
  \begin{array}{cccc}
   u_{11} & u_{12} & \cdots & u_{1n} \\
   0 & u_{22} & \cdots & u_{2n} \\
   \vdots & \vdots & \ddots & \vdots \\
   0 & 0 & \cdots & u_{nn}
  \end{array}
 \right]
\]

\[
a_{ij} = \sum_{k = 1}^{N}l_{ik} \, u_{kj}
=
 \left\{
  \begin{array}{ll}
   \displaystyle \sum_{k = 1}^{i}l_{ik} \, u_{kj}, & \mbox{se $i \leq j$}\\
   \displaystyle \sum_{k = 1}^{j}l_{ik} \, u_{kj}, & \mbox{se $i > j$}
  \end{array}
 \right.
\]

\begin{center}
$
\displaystyle a_{ij} = \sum_{k = 1}^{i-1}l_{ik} \, u_{kj} + l_{ii} \, u_{ij} \Rightarrow
$
\framebox{
$
\displaystyle  u_{ij} = a_{ij} - \sum_{k = 1}^{i-1} l_{ik} \, u_{kj}
$
}
\end{center}

\begin{center}
$
\displaystyle a_{ij} = \sum_{k = 1}^{j-1}l_{ik} \, u_{kj} + l_{ij} \, u_{jj} \Rightarrow
$
\framebox{
$
\displaystyle  l_{ij} = \frac{ a_{ij} - \displaystyle \sum_{k = 1}^{j-1} l_{ik} \, u_{kj} }{u_{jj}}
$
}
\end{center}

\noindent
\textbf{Resumo:}

\begin{enumerar}

\item Qualquer matriz não singular pode ser decomposta na forma $A = L \, U$.

\item Se um sistema de equações lineares tiver de ser resolvido repetidamente para múltiplos lados ?direitos?, a decomposição $L \, U$ é recomendada.

\item A matriz $U$ é idêntica a obtida no processo de eliminação de Gauss.

\item $L \, U$ é útil no cálculo do determinante.

\end{enumerar}
